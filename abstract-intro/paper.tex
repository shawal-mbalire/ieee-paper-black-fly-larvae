\documentclass[conference]{IEEEtran}
%\IEEEoverridecommandlockouts
% The preceding line is only needed to identify funding in the first footnote. If that is unneeded, please comment it out.
\usepackage{cite}
\usepackage{amsmath,amssymb,amsfonts}
\usepackage{algorithmic}
\usepackage{graphicx}
\usepackage{textcomp}
\usepackage{xcolor}
\def\BibTeX{{\rm B\kern-.05em{\sc i\kern-.025em b}\kern-.08em
    T\kern-.1667em\lower.7ex\hbox{E}\kern-.125emX}}


\begin{document}

    \title{Low cost IOT based cloud solution for Intelligence Integrated Black fly rearing.}

    \author{


        \IEEEauthorblockN{1\textsuperscript{st} Shawal Mbalire}
        \IEEEauthorblockA{\textit{School of Engineering,CEDAT} \\
        \textit{Makerere University}\\
        Kampala, Uganda \\
        mshawal49@icloud.com}


        \and
        \IEEEauthorblockN{2\textsuperscript{nd} Mary Patience Namugwanya}
        \IEEEauthorblockA{\textit{School of Engineering,CEDAT} \\
        \textit{Makerere University}\\
        Kampala, Uganda \\
        nmarypatience@gmail.com}


        \and
        \IEEEauthorblockN{3\textsuperscript{rd} Bradley Alfred Mpeera Tuwmine}
        \IEEEauthorblockA{\textit{School of Engineering,CEDAT} \\
        \textit{Makerere University}\\
        Kampala, Uganda \\
        bradleytumwine82@gmail.com}
    }

    \maketitle

    \begin{abstract}
       Black soldier fly(BSF), Hermetia illucens is a commonly and widespread fly of the family Stratiomyidae. Since the late 20th century, its larvae stage has increasingly gained attention due to its effective bio-waste conversion. Additionally, it has an excellent ability to transform waste into high quality protein. Used as alternative protein additives in animal feed, this translates into an inexpensive clean and sustainable food source. Though effective, it requires certain environmental conditions that needs to be monitored regularly to make sure the larvae can process the waste effectively. Additionally, several challenges remain to ensure that BSF farming is economically viable at different scales and can be widely implemented. Manual labor is required to ensure optimal conditions to rear the larvae, from aerating the feeding substrate to monitoring abiotic conditions during the growth cycle. But, BSF farmers are not always on site and the human resource is limited. Thus, an automated system with a remote monitoring capability is needed to ease the monitoring process. This paper proposes the Internet of Things (IoT) for environmental condition monitoring to ensure optimal growing conditions and data collection from the sensors to be obtained in real-time.
    \end{abstract}

    \begin{IEEEkeywords}
        black soldier fly ,iot
    \end{IEEEkeywords}

    \section{Introduction}
Bio-waste management is one of the main global challenges of our time, with significant repercussions on human and environmental health related to sanitary issues, pollution of ground water, and emission of greenhouse gases (GHGs). As global population and consumption rise, bio-waste production is also projected to increase significantly. Conventional methods of dealing with bio-waste, including open dumping in less economically developed countries or landfilling not equipped with means to capture GHGs such as methane, exacerbate the global warming crisis. Besides, landfills release various odors, attract disease vectors, and produce leachates that pollute ground water.
Black soldier fly (BSF), Hermetia illucens, is an insect that has gained popularity among other insect-based bio-waste treatments for its effectiveness to convert bio-waste using its larvae. Bio-waste biomass is consumed by the larvae, which then will be transformed into protein and fat of the larvae as well as its residue. The larvae is known to be having a high quality protein suitable for feeding chicken and fish, residue contains nutrients and organic matter that helps reduce soil depletion. Additionally, its effectiveness in reducing the risk of bacterial transmission through bio-waste makes BSF a safe option for bio-waste treatment in farm level.
Previously, BSF farming required wide spaces about 150 m2 that made it difficult to rear in locations with limited spaces like urban areas. Moreover, the rearing method needs special care to adjust suitable condition (temperature and humidity) and frequent attention from laborers which generally take times in each procedure.  For this reason, there was a need to establish an intelligent BSF rearing system.
The Internet of Things (IoT) has begun to emerge as it has helped human in controlling and monitoring essential conditions using devices which are able to capture, evaluate and transmit information from the environment to the cloud, where the data will be stored. Environment monitoring system is one of the most important IoT systems which mainly includes data collections through sensors and data reviewing for short-term measure as well as remote management and observations.
By using the IoT system to maintain the environmental condition i.e. temperature, relative humidity, light intensity and aeration of BSF larvae on its optimum level, data collection from the sensors needs to be obtained in real time to make sure that the data is accurate and representing the exact information of the environmental condition, as well as to be saved automatically to the database for certain period. This is to ensure that the user will have the proper information to interpret the insect’s environmental condition. Thus, data communication from the sensor to the database plays an important role in obtaining this need. 
This paper seeks to develop an automated solution to BSF rearing addressing two needs:
\begin{enumerate}
\item Controlling the environmental conditions, which is currently manual and labor-intensive. 
\item Remote monitoring of these conditions in real time.
\end{enumerate}




\end{document}
